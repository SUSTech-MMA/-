\section{微分方程}
定义:含有函数及导数的关系式称为微分方程。
\subsection{常微分方程}
一元函数及其导数。
$$\frac{dx}{dt}\,=\,f(x,t),\quad{}x(t_0)\,=\,x_0$$

\subsection{偏微分方程}
设$u$是$x,\,y$的二元函数,且满足
$$A\frac{\partial^2u}{\partial{}x^2}\,+\,B\frac{\partial^2u}{\partial{}x\partial{}y}\,+\,C\frac{\partial^2u}{\partial{}y^2}\,=\,F(x,y,u,\frac{\partial u}{\partial x},\frac{\partial u}{\partial y})$$
称为二阶常系数偏微分方程,记$D^2\,=\,B^2\,-\,4 A C$。
\begin{equation*}
\begin{cases}
D<0 & \text{椭圆方程,拉普拉斯方程:$u_{xx}+u_{yy}=0$}\\
D=0 & \text{抛物方程,热传导方程:$\frac{\partial u}{\partial t}=a\Delta u$},\, \\
D>0 & \text{双曲方程,波动方程:$\frac{\partial^2u}{\partial t^2}=a^2\Delta u$},\,
\end{cases}
\end{equation*}

\subsection{微分方程的求解}
\begin{itemize}
\item 理论上的分析
    \begin{itemize}
    \item 解是否存在,如果存在是否唯一,以及解的状态。
    \end{itemize}
\item 数值求解
    \begin{itemize}
    \item 常微分:欧拉格式,线性多步法,龙格库塔法等。
    \item 偏微分呢:差分法,有限元方法,谱方法等。
    \end{itemize}
\item MATLAB使用
    \begin{itemize}
    \item 符号计算工具箱,dsolve,pde工具箱
    \end{itemize}
\item 微分方程的例子
    \begin{itemize}
    \item SARS传播
    \end{itemize}
\end{itemize}

\subsection{SARS传播}
\subsubsection{问题的重述}
对SARS传播建立数学模型,为预测控制传染病的传播提供可靠足够的信息。
\subsubsection{问题的分析}
采用SIR模型(Susceptible Infected Recovered Model):\\
\phantom{may the force be with you}
\xymatrix{
\text{易感人群}\ar[d] & \text{感染者} \ar[d]\\
\text{潜伏人群}\ar@/_/[r]\ar[ru] & \text{移除者}
}

\subsubsection{模型假设与符号说明}
\begin{itemize}
\item 模型假设
    \begin{enumerate}
    \item 不考虑人口的出生率死亡率
    \item 假设平均潜伏期为6天(翻阅资料)
    \item 假设潜伏期的病人不传播
    \end{enumerate}
\item 符号说明
    \begin{itemize}
    \item 见表\ref{tab:symintro}。
    \end{itemize}
\end{itemize}

\begin{table}[htbp]
\centering
\begin{tabular}{@{}ll@{}}
\toprule
符号 & 说明 \\ \midrule
t & SARS的传播时间 \\
N & 某个疫区人口总数 \\
S(t) & t时刻健康人数占的比例 \\
E(t) & t时刻潜伏人数比例 \\
I(t) & 感染人数比例 \\
Q(t) & 移除者人数比例 \\
$\lambda$(t) & 日接触率 \\
b(t) & 每日新增确诊人数 \\
v(t) & 每日新增疑似病例 \\
d(t) & 每日新增死亡人数 \\
f(t) & 疫病指数 \\
g(t) & 预防措施力度 \\
h(t) & 个体警惕性指标 \\
w(t) & 防范意识 \\ \bottomrule
\end{tabular}
\caption{符号说明}\label{tab:symintro}
\end{table}


\subsubsection{模型的建立}
初步确定各类人群的关系:\xymatrix{
S \ar[r] & E \ar[r] & I \ar[r] & Q
}\\[1cm]\par
$t$时刻易感人群为$N\cdot{}S(t)$,$t+\Delta{}t$时刻易感人群为$N\cdot{}S(t+\Delta{}t)$。于是可以得到:
$$N\left(S(t)-S(t+\Delta{}t)\right)\,=\,N\cdot{}I(t)\cdot{}\lambda{}(t)\cdot{}S(t)\cdot{}\Delta{}t$$
令$\Delta{}t\rightarrow{}0$,则有:
$$-\frac{dS}{dt}\,=\,\lambda{}(t)\cdot{}S(t)\cdot{}I(t)$$\\[1cm]\par

同理可得$t$时刻潜伏人数关系式:
$$N\left(E(t)-E(t+\Delta{}t)\right)\,=\,N\left(I(t)\cdot{}\lambda{}(t)\cdot{}S(t)-\epsilon{}\cdot{}E(t)\right)\Delta{}t$$
令$\Delta{}t\rightarrow{}0$,则有:
$$-\frac{dE}{dt}\,=\,I(t)\cdot{}\lambda{}(t)\cdot{}S(t)-\epsilon{}\cdot{}E(t)$$\\[1cm]\par

同理可得$t$时刻死亡人数关系式:
$$N\left(Q(t+\Delta{}t)-Q(t)\right)\,=\,\omega{}\cdot{}N\cdot{}I(t)\Delta{}t$$
令$\Delta{}t\rightarrow{}0$,则有:
$$\frac{dQ}{dt}\,=\,\omega{}\cdot{}I(t)$$\\[1cm]\par

同理可得$t$时刻感染人数关系式:
$$N\left(I(t+\Delta{}t)-I(t)\right)\,=\,N\left(\epsilon\cdot E(t)-\omega\cdot I(t)\right)\Delta{}t$$
令$\Delta{}t\rightarrow{}0$,则有:
$$\frac{dI}{dt}\,=\,\epsilon\cdot E(t)-\omega\cdot I(t)$$\\[1cm]\par

通过查阅医学数据,取$\epsilon=\sfrac16$,$\omega=0.008$。\\[1cm]\par

控前模型,$\lambda{}(t)=\lambda_0$,同时存在超级传染事件发生,故引入$\delta$函数:$$\delta(x-x_0)=
\begin{cases}
\infty & if\,x=x_0 \\
0    & otherwise
\end{cases}$$
修正如上所得式子如下:
$$-\frac{dS}{dt}\,=\,\lambda{}(t)\cdot{}S(t)\cdot{}I(t)+N\cdot{}\sum_{i=1}^m\alpha_i\cdot\Delta{}(t-t_i)$$
$$-\frac{dE}{dt}\,=\,I(t)\cdot{}\lambda{}(t)\cdot{}S(t)+\epsilon{}\cdot{}E(t)+N\cdot{}\sum_{i=1}^m\alpha_i\cdot\Delta{}(t-t_i)$$
由此根据微分方程及初始条件($S(t_0)=S_0$等),代入数据即可得出结果。\\[1cm]\par

控后模型,消灭超级传染事件,并且$\lambda$(t)有所变化,具体符号说明见表\ref{tab:symintro2}。

\begin{table}[htbp]
\centering
\begin{tabular}{@{}ll@{}}
\toprule
符号 & 说明 \\ \midrule
b(t) & 每日新增确诊人数 \\
v(t) & 每日新增疑似病例 \\
d(t) & 每日新增死亡人数 \\
f(t) & 疫病指数 \\
g(t) & 预防措施力度 \\
h(t) & 个体警惕性指标 \\
w(t) & 防范意识 \\ \bottomrule
\end{tabular}
\caption{控后模型符号说明}\label{tab:symintro2}
\end{table}

可得控后模型的符号分析:\\
\phantom{what is dead may never die}
\xymatrix{
f(t)\ar[r]\ar@/_/[rd] & g(t)\ar[r] & w(t)\ar[r] & \lambda(t) \\
                  & h(t)\ar@/_/[ru]
}\\[1cm]\par

f(t)的刻画:\par
f(t)与d(t),b(t)与v(t)有关,所以构造线性组合(其中$q_1+q_2+q_3=1$):
$$f(t)\,=\,q_1\frac{d(t)}{max\lbrace{}d(t)\rbrace}+q_2\frac{b(t)}{max\lbrace{}b(t)\rbrace}+q_3\frac{v(t)}{max\lbrace{}v(t)\rbrace}$$
根据数据可以画出离散图形,根据图像观测为韦伯分布:
$$f(t)\,=\,\frac{m}{x_0}\cdot{}(t-v)^{m-1}\cdot{}\exp\Bigg(\dfrac{(t-v)^m}{x_0}\Bigg)$$
通过参数估计(矩估计,点估计),可得$m=2.3499$,$\,v=-1.5578$,$\,x_0=14.3530$。\\[1cm]\par

g(t)的刻画:\par
由$g(t_0)=k_0$变成与$f(t)$相关的函数,通过图像猜测为高斯分布,于是得到函数关系:
$$g(t)\,=\,k_0+k_1\cdot{}\Bigg(1-\exp(-\frac{\Big(\overline{f(t)}\Big)^2}{\sigma_1})\Bigg)$$
同样根据参数估计可得$k_0=0.2$,$\overline{f(t_0)}=0.42$,$g(t_0)=0.7$。\\[1cm]\par

h(t)的刻画:
\begin{equation*}
\begin{cases}
f(t_0)=0 & h(t_0)=k_2 \\
f(t_0)\rightarrow \infty & h(t_0)\rightarrow 1
\end{cases}
\end{equation*}
于是可得$h(t)\,=\,k_2-k_3\cdot{}\exp(-f(t))$,通过如上关系式可得$k_2=1$,$k_3=0.8$。\\[1cm]\par

$\lambda$(t)的刻画:
\begin{equation*}
\begin{cases}
w(t_0)=0 & \lambda(t_0)=\lambda_0 \\
w(t)\quad inc & \lambda(t)\quad dec
\end{cases}
\end{equation*}
于是有:
$$\lambda(t)\,=\,k_4\cdot\Bigg(1-\exp\Big(-\frac{(1-w(t))^2}{\sigma_2}\Big)\Bigg)$$

\subsubsection{模型的求解}
控后的模型方程组即为在控前模型上增加$\lambda$(t)的约束条件,求解即为常微分方程的求解,可以使用欧拉格式。

\subsubsection{模型的分析}
可继续从如下两个关系式,每日新增确诊人数与疑似病例中对控前控后的模型进行分析。
\begin{equation*}
\begin{cases}
N\Big(I(t)-I(t-1)\Big) &=\quad b(t) \\
N\Big(E(t)-E(t-1)\Big) &=\quad v(t)2
\end{cases}
\end{equation*}