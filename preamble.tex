
\usepackage{enumerate,amsmath,amssymb,fancyhdr,tikz,tikzpagenodes,xypic,booktabs,xfrac,graphicx,float,multirow,makecell,xcolor,pifont}
\usepackage[all,pdf]{xy}
\usetikzlibrary{shapes.geometric}
\pagestyle{fancy}
\fancyhf{}
\renewcommand\headrulewidth{0pt}
\fancyhead[OC]{\begin{tikzpicture}[remember picture,overlay]
\node[diamond,draw,font=\small\itshape] at (current page header area.south west) (dia) {\thepage};
\draw (dia.3) -- (current page header area.south east|-dia.3);
\draw (dia.357) -- ([xshift=-7pt]current page header area.south east|-dia.357);
\end{tikzpicture}}
\fancyhead[EC]{\begin{tikzpicture}[remember picture,overlay]
\node[diamond,draw,font=\small\itshape] at (current page header area.south east) (dia) {\thepage};
\draw (dia.177) -- (current page header area.south west|-dia.177);
\draw (dia.183) -- ([xshift=7pt]current page header area.south west|-dia.183);
\end{tikzpicture}}
\fancyhead[OR]{\small\nouppercase\leftmark}
\fancyhead[EL]{\small\nouppercase\rightmark}

\title{数学建模课程笔记}
\author{Iydon,张志炅}
% depth of section
\setcounter{tocdepth}{3}
% example
\usepackage{environ}
\usepackage{xparse}
\usetikzlibrary{shapes,decorations}
\definecolor{seco}{RGB}{0,145,215}
\newcommand{\newfancytheoremstyle}[5]{%
    \tikzset{#1/.style={draw=#3, fill=#2,very thick,rectangle, rounded corners, inner sep=10pt, inner ysep=20pt}}
    \tikzset{#1title/.style={fill=#3, text=#2}}
    \expandafter\def\csname #1headstyle\endcsname{#4}
    \expandafter\def\csname #1bodystyle\endcsname{#5}
}
\newfancytheoremstyle{fancythrm}{white!10}{seco}{\bfseries\sffamily}{\sffamily}
\makeatletter
\DeclareDocumentCommand{\newfancytheorem}{ O{\@empty} m m m O{fancythrm} }{%
    % define the counter for the theorem
    \ifx#1\@empty
        \newcounter{#2}
    \else
        \newcounter{#2}[#1]
        \numberwithin{#2}{#1}
    \fi
    %% define the "newthem" environment
    \NewEnviron{#2}[1][{}]{%
        \noindent\centering
        \begin{tikzpicture}
            \node[#5] (box){
                \begin{minipage}{0.93\columnwidth}
                    \csname #5bodystyle\endcsname \BODY~##1
                \end{minipage}};
            \node[#5title, rounded corners, right=10pt] at (box.north west){
                {\csname #5headstyle\endcsname #3 \stepcounter{#2}\csname the#2\endcsname\; ##1}};
            \node[#5title, rounded corners] at (box.east) {#4};
        \end{tikzpicture}
    }[\par\vspace{.5\baselineskip}]
}
\makeatother
\newfancytheorem[section]{example}{例}{$\pi$}